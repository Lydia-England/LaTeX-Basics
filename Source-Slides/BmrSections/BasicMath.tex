\begin{frame}[fragile]
\frametitle{Basic Math Notation with \LaTeX{}}
Typesetting mathematical expressions is one of \LaTeX{}'s greatest strengths --- and one of the primary motivations which drove Donald Knuth to develop the original TeX system. \\[\baselineskip]
\LaTeX{} provides thousands of different mathematical symbols and supports multiple different math environments. \\[\baselineskip]
\LaTeX{} is the mathematical typesetting standard in all technical disciplines and in many related fields.
\end{frame}

\begin{frame}[fragile]
\frametitle{Math Modes}
\begin{columns}
\column{0.5\textwidth}
\begin{center}
    \textbf{Display Style Math Mode} \\
    \textit{Puts math on display} \\
    \vspace{0.4cm}
    \fbox{\parbox{\textwidth}{\small It follows that
    \begin{align}
        T(\Vec{u}) & = \sum^n_{j=1} \sum^m_{k=1} M_{jk} x_k \Vec{w}_j 
    \end{align}
    Examination of this formula shows that
    \[y_j = \sum^m_{k=1} M_{jk} x_k\]
    }}
\end{center}

\column{0.5\textwidth}
\begin{center}
    \textbf{Inline (text) Style Math Mode} \\
    \textit{Math stays in the line} \\
    \vspace{0.4cm}
     \fbox{\parbox{\textwidth}{\small But if $T: V \to W$ is a linear transformation then $T(\vec{u}) = T(x_1 \Vec{v}_1 + x_2 \Vec{v_2} + ... + x_m \Vec{m})$. 
     Moreover, because we know that \(T(\Vec{v}_1), T(\Vec{v}_2),...,T(\Vec{v}_m)\) are the elements of the vector space \(W\), we can therefore write 
        \(T(\Vec{v}_k) = \sum^n_{j=1} M_{jk} \Vec{w}_j \) where the quantities \(M_{jk}\) are scalars. \\[\baselineskip]
    }}
\end{center}
\end{columns}
\end{frame}



\begin{frame}[fragile]
\frametitle{Math Modes: Display} 
% \textbf{Display Style Math Mode} \\
\begin{block}{Delimiters for Display Math Mode}
\small
Primary delimiters: \verb|\begin{displaymath} ... \end{displaymath}| \\
Short form \LaTeX{} delimiters: \verb|\[ ... \]| \\
Short form TeX delimiters: \verb|$$ ... $$| \\
\end{block}
\begin{exampleblock}{}
\small
\begin{minted}{latex}
    \[ \iiint\limits_S dV 
        =   \int^{2\pi}_{\phi=0}
            \int^{\pi}_{\theta=0}  
            \int^{r}_{\rho=0} 
            \rho^2 d\cos{\theta} d\phi \]
\end{minted}
    \[ \iiint\limits_S dV = \int^{2\pi}_{\phi=0} \int^{\pi}_{\theta=0} \int^{r}_{\rho=0} \rho^2 d\cos{\theta} d\phi \]
\end{exampleblock}
\end{frame}


\begin{frame}[fragile]
\frametitle{Math Modes: Inline} 
% \textbf{Display Style Math Mode} \\
\begin{block}{Delimiters for Inline Math Mode}
\small
Primary delimiters: \verb|\begin{math} ... \end{math}| \\
Short form \LaTeX{} delimiters: \verb|\( ... \)| \\
Short form TeX delimiters: \verb|$ ... $| \\
\end{block}
% \textbf{Example:} 
\vspace{0.3cm}
\small
\begin{minted}{latex}
\( p(x) = a_n x^n + a_{n-1} x^{n-1} + a_{n-2} x^{n-2} 
        + ... + a_2 x^2 + a_1 x + a_0 \)

$c^2 = a^2 + b^2 - 2ab\cos{\theta}$
\end{minted}
\vspace{0.3cm}
\( p(x) = a_n x^n + a_{n-1} x^{n-1} + a_{n-2} x^{n-2} 
        + ... + a_2 x^2 + a_1 x + a_0 \) \\[\baselineskip]
\vspace{0.3cm}
$c^2 = a^2 + b^2 - 2ab\cos{\theta}$
\end{frame}


\begin{frame}[fragile]
\frametitle{Math Modes: Which one should you use?}
\begin{columns}
    \column{0.5\linewidth}
    \textbf{Use display style math when...}
    \begin{itemize}
        \item[$\bullet$] The mathematical expression is set apart from the main text
        \item[$\bullet$] You want to number an equation (and possibly label and reference it later in the document)
    \end{itemize}
    \column{0.5\linewidth}
    \textbf{Use inline style math when...}
    \begin{itemize}
        \item[$\bullet$] The mathematical expression is included within a line or paragraph of text
        \item[$\bullet$] You are referencing a variable in a line of text (i.e., ``We have the variables $x$, $y$, and $z$)
    \end{itemize}
\end{columns}
\begin{block}{Change style within mode}
Force display math in inline mode: \verb|\( \displaymath ... \)| \\
Force inline style math in display mode: \verb|\[ \textstyle ... \]| \\
\end{block}
\end{frame}


\begin{frame}[fragile]
\frametitle{Math Modes: Examples}
\begin{exampleblock}{Display Style / Inline Mode}
    \small The Basel problem is the problem of computing the sum of the squares of the reciprocals of natural numbers. It turns out that \( \displaystyle \sum^\infty_{n=1} \frac{1}{n^2} = \frac{\pi^2}{6}\). In 1735, Euler proved this result.
\end{exampleblock}
\begin{exampleblock}{Inline Style / Inline Mode}
    \small The Basel problem is the problem of computing the sum of the squares of the reciprocals of natural numbers. It turns out that \( \sum^\infty_{n=1} \frac{1}{n^2} = \frac{\pi^2}{6}\). In 1735, Euler proved this result.
\end{exampleblock}
\end{frame}

\begin{frame}[fragile]
\frametitle{Math Modes: Examples}
\begin{exampleblock}{Inline Style / Display Mode}
    \small The Basel problem is the problem of computing the sum of the squares of the reciprocals of natural numbers. It turns out that \[ \textstyle \sum^\infty_{n=1} \frac{1}{n^2} = \frac{\pi^2}{6}\] In 1735, Euler proved this result.
\end{exampleblock}
\begin{exampleblock}{Display Style / Display Mode}
    \small The Basel problem is the problem of computing the sum of the squares of the reciprocals of natural numbers. It turns out that \[ \sum^\infty_{n=1} \frac{1}{n^2} = \frac{\pi^2}{6}\] In 1735, Euler proved this result.
\end{exampleblock}
\end{frame}


\begin{frame}[fragile]
\frametitle{Math Modes: \texttt{amsmath} Package}
\begin{itemize}
    \item[$\bullet$] Developed for the American Mathematical Society (AMS)
    \item[$\bullet$] Principle package of the AMS-\LaTeX{} distribution
    \item[$\bullet$] Provides additional features to facilitate mathematical typesetting
    \item[$\bullet$] \textit{You should be using this package in any document with mathematical notation} 
    \item[$\bullet$] \verb|\usepackage{amsmath}|
\end{itemize}
% \begin{centering}
% \includegraphics[width=0.5\linewidth]{img/ams_logo.png}
% \end{centering}
\tikz[remember picture, overlay] {\node[anchor=south east, outer sep=15pt] at (current page.south east) {\includegraphics[width=5cm]{img/ams_logo.png}};}
\end{frame}



\begin{frame}[fragile]
\frametitle{Math Modes: \texttt{amsmath} Package} 
\begin{block}{Writing a Single Equation}
    \verb|\begin{equation} ... \end{equation}| \\
\end{block}
To write an equation, use the \verb|\equation| or \verb|\*equation| environment depending on if you want your equation numbered. % Additionally, labels can be added with \verb|\label{}| to reference the equation later within the document.
\begin{exampleblock}{Example}
\small
\begin{minted}{latex}
\begin{equation} \label{eu_eqn}
e^{\pi i} + 1 = 0
\end{equation}

The beautiful equation \ref{eu_eqn} is known as the Euler
equation.
\end{minted}
\begin{equation} \label{eu_eqn}
e^{\pi i} + 1 = 0
\end{equation}

The beautiful equation~\ref{eu_eqn} is known as the Euler
equation.
\end{exampleblock}
\end{frame}



\begin{frame}[fragile]
\frametitle{Math Modes: \texttt{amsmath} Package} 
\begin{block}{Displaying Long Equations}
    \verb|\begin{multline} ... \end{multline}| \\
\end{block}
For equations longer than a line, use the \verb|multline| or \verb|multline*| environment. Use a double backslash to indicate where the equation should be broken.
\begin{exampleblock}{Example}
\small
\begin{minted}{latex}
\begin{multline*}
p(x) = 3x^6 + 14x^5y + 590x^4y^2 + 19x^3y^3\\ 
- 12x^2y^4 - 12xy^5 + 2y^6 - a^3b^3
\end{multline*}
\end{minted}
\begin{multline*}
p(x) = 3x^6 + 14x^5y + 590x^4y^2 + 19x^3y^3\\ 
- 12x^2y^4 - 12xy^5 + 2y^6 - a^3b^3
\end{multline*}
\end{exampleblock}
\end{frame}


\begin{frame}[fragile]
\frametitle{Math Modes: \texttt{amsmath} Package} 
\begin{block}{Splitting and Aligning an Equation }
\small
\verb|\begin{equation}\begin{split} ... \end{split}\end{equation}| \\
\end{block}
\verb|Split| is similar to \verb|multline|. It is used to break an equation and align it in columns. It is used within an \verb|equation| environment. 
\begin{exampleblock}{Example}
\small
\begin{minted}{latex}
\begin{equation} \begin{split}
A & = \frac{\pi r^2}{2} \\
 & = \frac{1}{2} \pi r^2
\end{split} \end{equation}
\end{minted}
\begin{equation} \begin{split}
A & = \frac{\pi r^2}{2} \\
 & = \frac{1}{2} \pi r^2
\end{split} \end{equation}
\end{exampleblock}
\end{frame}


\begin{frame}[fragile]
\frametitle{Math Modes: \texttt{amsmath} Package} 
\begin{block}{Aligning Several Equations}
\small
\verb|\begin{align} ... \end{align}| \\
\end{block}
To align several equations vertically, use the \verb|align| environment. The ampersand character \& determines where equations align.
\begin{exampleblock}{Example}
\small
\begin{minted}{latex}
\begin{align*}
x&=y           &  w &=z              &  a&=b+c\\
-4 + 5x&=2+y   &  w+2&=-1+w          &  ab&=cb
\end{align*}
\end{minted}
\begin{align*}
x&=y           &  w &=z              &  a&=b+c\\
-4 + 5x&=2+y   &  w+2&=-1+w          &  ab&=cb
\end{align*}
\end{exampleblock}
\end{frame}


\begin{frame}[fragile]
\frametitle{Math Modes: \texttt{amsmath} Package} 
\begin{block}{Grouping and Centering Equations}
\small
\verb|\begin{align} ... \end{align}| \\
\end{block}
To display a set of consecutive equations, centered, with no alignment, use the \verb|gather| environment.
\begin{exampleblock}{Example}
\small
\begin{minted}{latex}
\begin{gather*} 
2x - 5y =  8 \\ 
3x^2 + 9y =  3a + c
\end{gather*}
\end{minted}
\begin{gather*} 
2x - 5y =  8 \\ 
3x^2 + 9y =  3a + c
\end{gather*}
\end{exampleblock}
\end{frame}


\begin{frame}[fragile]
\frametitle{Math Modes: \texttt{amsmath} Package} 
% \textbf{Display Style Math Mode} \\
\begin{block}{Environments for Display Math Mode (Numbered)}
\small
\verb|\begin{equation} ... \end{equation}| \\
\verb|\begin{align} ... \end{align}| \\
\verb|\begin{gather} ... \end{gather}| \\
\verb|\begin{multline} ... \end{multline}| \\
\end{block}
\begin{exampleblock}{}
\small
\begin{minted}{latex}
\begin{equation}
    \iiint\limits_S dV 
        =   \int^{2\pi}_{\phi=0} 
            \int^{\pi}_{\theta=0}
            \int^{r}_{\rho=0} 
            \rho^2 d\cos{\theta} d\phi
\end{equation}
\end{minted}
    \begin{equation}
        \iiint\limits_S dV 
        =   \int^{2\pi}_{\phi=0} 
            \int^{\pi}_{\theta=0} 
            \int^{r}_{\rho=0} 
            \rho^2 d\cos{\theta} d\phi
    \end{equation}
\end{exampleblock}
\end{frame}


\begin{frame}[fragile]
\frametitle{Math Modes: \texttt{amsmath} Package} 
% \textbf{Display Style Math Mode} \\
\begin{block}{Environments for Display Math Mode (Unnumbered)}
\small
\verb|\begin{equation*} ... \end{equation*}| \\
\verb|\begin{align*} ... \end{align*}| \\
\verb|\begin{gather*} ... \end{gather*}| \\
\verb|\begin{multline*} ... \end{multline*}| \\
\end{block}
\begin{exampleblock}{}
\small
\begin{minted}{latex}
\begin{equation*}
    \iiint\limits_S dV 
        =   \int^{2\pi}_{\phi=0} 
            \int^{\pi}_{\theta=0}
            \int^{r}_{\rho=0} 
            \rho^2 d\cos{\theta} d\phi
\end{equation*}
\end{minted}
    \begin{equation*}
        \iiint\limits_S dV 
        =   \int^{2\pi}_{\phi=0} 
            \int^{\pi}_{\theta=0} 
            \int^{r}_{\rho=0} 
            \rho^2 d\cos{\theta} d\phi
    \end{equation*}
\end{exampleblock}
\end{frame}




\begin{frame}[fragile]
\frametitle{Subscripts and Superscripts}
\begin{block}{Subscripts}
    Subscripts are written as \verb|a_b|, displaying as $a_b$. 
\end{block}
\begin{block}{Superscripts}
    Superscripts are written as \verb|a^b|, displaying as $a^b$.
\end{block}
\begin{block}{Combining Superscripts and Subscripts}
    Superscripts and subscripts can be combined and nested:
    \begin{itemize}
    \item \small \verb| T^{i_1 i_2 \dots i_p}_{j_1 j_2 \dots j_q} | \\
    \verb| = T(x^{i_1},\dots,x^{i_p},e_{j_1},\dots,e_{j_q}) | 
    \\ displays as \[ T^{i_1 i_2 \dots i_p}_{j_1 j_2 \dots j_q} = T(x^{i_1},\dots,x^{i_p},e_{j_1},\dots,e_{j_q}) \] 
    \end{itemize}
\end{block}
\end{frame}


\begin{frame}[fragile]
\frametitle{Integrals and Summations}
\begin{block}{Integrals}
    Integrals are written as \verb|\int|, displaying as $\int$. \\
    Closed integrals are written as \verb|\oint|, displaying as $\oint$. \\
    Limits are placed on integrals with superscripts and subscripts.\\
    \textit{Example:} \verb|\int_a^b x^2 dx| displays as $\int_a^b x^2 dx$.\\
\end{block}
\begin{block}{Summations}
    Summations are written as \verb|\sum|, displaying as $\sum$. \\
    \textit{Limits (inline):} \verb|\(\sum_{n=1}^{\infty})\| displays as $\sum_{n=1}^{\infty}$.\\
    \textit{Limits (display):} \verb|\[\sum_{n=1}^{\infty}]\| displays as \[\sum_{n=1}^{\infty}\]
\end{block}
\end{frame}


\begin{frame}[fragile]
\frametitle{Fractions}
    \begin{block}{Fractions}
    Fractions can be written several different ways:
    \begin{itemize}
    \item Inline mode, without fraction formatting:
        \begin{itemize}
        \item[$\bullet$] \verb|\(p(x) = (a x^2 + b x)/(d x + c)\)| 
        \item displays as $p(x) = (a x^2 + b x)/(d x + c)$
        \end{itemize}
    \item Inline mode, with fraction formatting:
        \begin{itemize}
        \item[$\bullet$] \verb|\( p(x) = \frac{a x^2 + b x}{d x + c} \)| 
        \item displays as \( p(x) = \frac{a x^2 + b x}{d x + c}\)
        \end{itemize}
    \item Display mode, with fraction formatting:
        \begin{itemize}
        \item[$\bullet$] \verb|\[ p(x) = \frac{a x^2 + b x}{d x + c}\]| 
        \item displays as 
        \end{itemize}
    \item \[ p(x) = \frac{a x^2 + b x}{d x + c}\]
    \end{itemize}
\end{block} 
\end{frame}


\begin{frame}[fragile]
\frametitle{Greek Letters, Arrows}
\begin{block}{Greek Letters}
    Lower case Greek letters are written as:
    \begin{itemize}
        \item \verb|$\omega$| (displaying as $\omega$), 
        \item \verb|$\delta$| (displaying as $\delta$), etc.
    \end{itemize}
    While upper case Greek letters are written as 
    \begin{itemize}
        \item \verb|$\Omega$| (displaying as $\Omega$), 
        \item \verb|$\Delta$| (displaying as $\Delta$), etc.
    \end{itemize}
\end{block}  
\begin{block}{Arrows (non-exhaustive list)}
    \begin{columns}
    \column{0.1\linewidth}
        \verb|\leftarrow| 
        \verb|\rightarrow|
        \verb|\uparrow|
        \verb|\Uparrow|
        \verb|\mapsto|
    \column{0.1\linewidth} 
        $~~\leftarrow$ \\
        $~~\rightarrow$\\
        $~~\uparrow$\\
        $~~\Uparrow$\\
        $~~\mapsto$
    \column{0.1\textwidth}
        \verb|\Leftarrow| 
        \verb|\Rightarrow|
        \verb|\downarrow|
        \verb|\Downarrow|
        \verb|\longmapsto|
     \column{0.1\linewidth} 
        $~~\Leftarrow$ \\
        $~~\Rightarrow$\\
        $~~\downarrow$\\
        $~~\Downarrow$\\
        $~~\longmapsto$   
    \end{columns}
\end{block}
\end{frame}


\begin{frame}[fragile]
\frametitle{Math Symbols}
The \texttt{amssymb} package provides many common mathematical symbols. \\
    \begin{block}{Math Symbols (non-exhaustive list)}
    \begin{columns}
    \column{0.1\linewidth}
        \verb|\infty| 
        \verb|\Re|
        \verb|\forall|
        \verb|\nabla|
        \verb|\partial|
        \verb|\square|
    \column{0.1\linewidth} 
        $~~$ \infty \\
        $~~$ \Re \\
        $~~$ \forall \\
        $~~$ \nabla \\
        $~~$ \partial \\
        $~~$ \square
    \column{0.1\textwidth}
        \verb|\triangle| 
        \verb|\Im|
        \verb|\exists|
        \verb|\nexists|
        \verb|\cdots|
        \verb|\blacksquare|
     \column{0.1\linewidth} 
        $~~\triangle$ \\
        $~~\Im$ \\
        $~~\exists$ \\
        $~~\nexists$ \\
        $~~\cdots$ \\
        $~~\blacksquare$
    \end{columns}
\end{block}
\end{frame}


\begin{frame}[fragile]
\frametitle{Operators}
    Mathematical operators are prefixed with a backslash. Arguments can be delimited with curly braces, but don't have to be.
\begin{block}{Operators (non-exhaustive list)}
    \begin{columns}
    \column{0.15\linewidth}
        \verb|\sin{x}| \\
        \verb|\cos{x}| \\
        \verb|\tan{x}| \\
        \verb|\sec{x}| \\
        \verb|\sinh | \\
        \verb|\det(A)| \\
        \verb|\lim_{h\to 0}| \\
        \verb|\min(a, b)| \\
        \verb|\ln(x)| \\
    \column{0.15\linewidth} 
        $~~ \sin{x} $ \\
        $~~ \cos{x} $ \\
        $~~ \tan{x} $ \\
        $~~ \sec{x} $ \\
        $~~ \sinh x $ \\
        $~~ \det(A)$\\
        $~~ \lim_{h \to 0 } $ \\
        $~~ \min(a,b)$ \\
        $~~ \ln(x)$ \\
    \column{0.15\textwidth}
        \verb|\arcsin x| \\
        \verb|\arccos x| \\
        \verb|\arctan x| \\
        \verb|\cot x| \\
        \verb|\cosh x| \\
        \verb|\coth x| \\
        \verb|\ker{A}| \\
        \verb|\max(a,b) | \\
        \verb|\log(x) | \\
     \column{0.15\linewidth} 
        $~~ \arcsin x $ \\
        $~~ \arccos x $ \\
        $~~ \arctan x $ \\
        $~~ \cot x $ \\
        $~~ \cosh x $ \\
        $~~ \coth x $ \\
        $~~ \ker{A} $ \\
        $~~ \max(a,b)$ \\
        $~~ \log(x) $
    \end{columns}
\end{block}
Operators without arguments should include empty curly braces (like \verb|\arcsin{}|) or should be followed by a tilde (like \verb|\sinh~|).
\end{frame}


\begin{frame}[fragile]
\frametitle{Basic Math Notation: Essential Packages}
\begin{itemize}
    \item[$\bullet$] \texttt{amsmath} enhances math mode, provides symbols, math environments, alignment options, and more.
    \item[$\bullet$] \texttt{amssymb} provides extra math symbols.
    \item[$\bullet$] \texttt{mathtools} is based on \texttt{amsmath}; it fixes several deficiencies.
    \item[$\bullet$] \texttt{bm} provides support for boldface in math mode.
\end{itemize}
\end{frame}


\begin{frame}[fragile]
\frametitle{More fun math mode things!}
    \begin{itemize}
        \item[$\bullet$] Matrices (\verb|\vmatrix, \pmatrix, \bmatrix, \matrix|)
        \item[$\bullet$] Parenthesis sizing with \verb|\left(|, \verb|\right)|, big, bigg, Bigg, etc.
        \item[$\bullet$] \verb|\varepsilon| and \verb|\varphi|
        \item[$\bullet$] \verb|\varnothing| vs. \verb|\emptyset|
        \item[$\bullet$] Binomials (\verb|\binom{n}{k}|)
        \item[$\bullet$] Hats, bars, and vectors 
            (\verb|\hat{i}|, \verb|\widehat{G}|, \verb|\Bar{p}, \overline{lmnop}, \Vec{v}|)
        \item[$\bullet$] Non-breaking space with tilde (\verb|~|)
        \item[$\bullet$] Labels and internal references 
    \end{itemize}
\end{frame}


\begin{frame}[fragile]
\frametitle{Basic Math Notation: Summary}
\LaTeX{} has two math modes/styles, each having several delimiter options: 
\begin{itemize}
    \item Inline (for math expressions within lines of text)
    \item Display (for separate math expressions)
\end{itemize} \pause
The most common \LaTeX{} package for math expressions is \texttt{amsmath}:
\begin{itemize}
    \item Options: Equation, Split, Multline, Align, Gather
\end{itemize} \pause
Mathematical expressions can include:
\begin{itemize}
    \small
    \item Superscripts, Subscripts (\verb|a^b, a_b, a^{spr}, b_{sub}|)
    \item Integrals, Summations (\verb|\int, \sum, \int_a^b, \sum_{x=1}^n|)
    \item Fractions (\verb|\frac{numerator}{denominator}|)
    \item Symbols (Greek letters, arrows, etc.)
    \item Operators (\verb|\operator{argument}|)
\end{itemize}
\end{frame}