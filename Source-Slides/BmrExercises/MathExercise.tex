
% {\setbeamercolor{background canvas}{bg=red!15}

\begin{frame}[fragile]
\frametitle{Exercise: Mathematical Notation}
Add the following lines to the preamble of your document. \\
\textit{\small If you write a lot of documents with math, you should load these packages in every project!} \\
\begin{alertblock}{Math Packages}
    \small
    \begin{minted}{latex}
    \usepackage{amsmath}
    \usepackage{amssymb}
    \usepackage{mathtools}
    \usepackage{bm}
    \end{minted}
\end{alertblock}
\end{frame}


\begin{frame}[fragile]
\frametitle{Challenge: Proof of Euler Identity} 
\begin{alertblock}{Proof of Euler Identity}
    \textbf{Proof:} Consider the function \( f(t) = e^{-it} \left( \cos{t} + i \sin{t} \right) \) for \(t \in \Re \). 
    By the product rule
    \begin{equation}\tag{1}
        f'(t) 
            = e^{-it} \left(i \cos{t} 
            - \sin{t} \right) 
            - i e^{-it} \left( \cos{t} 
            + i \sin{t} \right) 
        = 0
    \end{equation}
    identically for all \( t \in \Re \). Hence, $f$ is constant everywhere. Since \( f(0) = 1 \), it follows that \( f(t) = 1 \) identically. Therefore, \( e^{it} = \cos{t} + i \sin{t} \) for all \( t \in \Re \). \(\blacksquare\)
\end{alertblock}
\end{frame}


\begin{frame}[fragile]
\frametitle{Challenge: Proof of Euler Identity} 
\begin{alertblock}{Proof of Euler Identity: Source Code}
    \small
    \begin{minted}{latex}
    \textbf{Proof:} Consider the function \( f(t) 
        = e^{-it} \left( \cos{t} 
        + i \sin{t} \right) \) for \(t \in \Re \). 
    By the product rule
    \begin{equation}
        f'(t) = 
            e^{-it} \left(i \cos{t} - \sin{t} \right) 
            - i e^{-it} \left( \cos{t}  
            + i \sin{t} \right) = 0
    \end{equation}
    identically for all \( t \in \Re \). Hence, $f$ is 
    constant everywhere. Since \( f(0) = 1 \), it follows 
    that \( f(t) = 1 \) identically. 
    Therefore, \( e^{it} = \cos{t} + i \sin{t} \) for 
    all \( t \in \Re \). \(\blacksquare\)
    \end{minted}
\end{alertblock}
\end{frame}


\begin{frame}[fragile]
\frametitle{Challenge: Laplacian in Spherical Coordinates} 
\begin{alertblock}{Laplacian in Spherical Coordinates}
    \begin{gather} \tag{2}
    \nabla^2 f 
    = \frac{1}{r} \frac{\partial^2}{\partial r^2} \left( r f \right) 
    + \frac{1}{r^2 \sin{\theta}} \frac{\partial}{\partial \theta} \left( \sin{\theta} \frac{\partial f}{\partial \theta} \right) 
    + \frac{1}{r^2 \sin^2{\theta}} \frac{\partial^2 f}{\partial \phi^2} \\ \notag
    \end{gather}
\end{alertblock}
\end{frame}


\begin{frame}[fragile]
\frametitle{Challenge: Laplacian in Spherical Coordinates} 
\begin{alertblock}{Laplacian in Spherical Coordinates: Source Code}
    \small
    \begin{minted}{latex}
    \begin{gather}
    \nabla^2 f 
    = \frac{1}{r} \frac{\partial^2}{\partial r^2} 
    \left( r f \right) 
    + \frac{1}{r^2 \sin{\theta}} 
    \frac{\partial}{\partial \theta} 
    \left( \sin{\theta} 
    \frac{\partial f}{\partial \theta} \right) 
    + \frac{1}{r^2 \sin^2{\theta}} 
    \frac{\partial^2 f}{\partial \phi^2}
    \end{gather}
    \end{minted}
\end{alertblock}
\end{frame}


\begin{frame}[fragile]
\frametitle{Challenge: Divergence Theorem} 
\begin{alertblock}{Divergence Theorem}
    \begin{gather} \tag{3}
    \iiint_v \left( \Vec{\nabla} \cdot \Vec{F} \right) dV = \oiint_s \left( \Vec{F} \cdot \hat{n} \right) dS \\ \notag
    \end{gather}
\end{alertblock}
\end{frame}


\begin{frame}[fragile]
\frametitle{Challenge: Divergence Theorem} 
% \begin{alertblock}{}
%     \small
%     \begin{minted}{latex}
%     \usepackage{esint}
%     \end{minted}
% \end{alertblock}
\begin{alertblock}{Divergence Theorem: Source Code}
    \small
    \begin{minted}{latex}
    \usepackage{esint}
    
    \begin{gather}
    \iiint_v \left( \Vec{\nabla} \cdot \Vec{F} \right) dV 
    = \oiint_s \left( \Vec{F} \cdot \hat{n} \right) dS 
    \end{gather}
    \end{minted}
\end{alertblock}
\end{frame}



% }
