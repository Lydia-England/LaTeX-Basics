\begin{frame}[fragile]
\frametitle{Figures}
    \begin{wrapfigure}{r}{0.4\textwidth}
        \centering
        \includegraphics[width=0.35\textwidth]{img/xkcd1301_file_extensions.png}
        \caption{File Extensions}
        \label{fig:xkcd1301}
    \end{wrapfigure}
    Images are essential elements of scientific documents. \\
    Thankfully, \LaTeX{} makes them easy to manipulate with tools like... \pause
    \begin{itemize}[$\bullet$]
        \item Figures and subfigures \pause
        \item Sizing and rotating \pause
        \item Positioning (absolute or relative) \pause
        \item Captioning, labelling, and cross-referencing \pause
        \item  List of Figures
    \end{itemize}
\end{frame}
%%% wrapfigure: this command accepts two mandatory arguments: the first one is to select where we want the image, it can be r or l, that is, right or left; the second one is the width to be reserved for the image


\begin{frame}[fragile]
\frametitle{Figures: The Graphicx Package}
\LaTeX{} does not manage images by itself, so we include the \keytt{graphicx} package. \\ \pause
The \keytt{graphicx} package provides the ability to reference files from filepaths within a project. \\[\baselineskip] \pause
To load the package (in the preamble):
\begin{exampleblock}{}
    \begin{minted}{latex}
    \usepackage{graphicx}
    \end{minted}
\end{exampleblock} \pause
\vspace{0.2cm}
The most useful commands in this package are the \texttt{includegraphics} command and the \texttt{graphicspath} command:
\begin{exampleblock}{}
    \begin{minted}{latex}
    \includegraphics[keys]{filename}
    \graphicspath{ {path} }
    \end{minted}
\end{exampleblock}
\end{frame}


\begin{frame}[fragile]
\frametitle{Figures: Folders and Paths}
It is good practice to keep images in one or more separated folders so your project is organized.\\ \pause
You can tell \LaTeX{} where to look for images... \pause
\begin{block}{\small Path relative to current working directory}
\small \mintinline{latex}{ \graphicspath{ {img/} }} tells \LaTeX{} to look for images in the \texttt{img} folder relative to the current working directory.
\end{block} \pause
\begin{block}{\small Path relative to \texttt{main} file}
\small \mintinline{latex}{\graphicspath{ {./img/} }} tells \LaTeX{} to look for images in the \texttt{img} folder relative to the \texttt{main} file (denoted as \texttt{./} ).
\end{block} \pause
\small \textit{The path can also be \emph{absolute} if the exact location on your file system is specified.} \\ \pause
\textit{Or you can skip the declaration and reference the filepath directly every time instead.}
\end{frame}


\begin{frame}[fragile]
\frametitle{Figures: Positioning}

\end{frame}


\begin{frame}[fragile]
\frametitle{Figures: Positioning}
\begin{columns}
    \column{0.6\linewidth}
    The \keytt{figure} environment provides automatic positioning. 
    It uses a positioning parameter passed inside brackets. \\%\\[\baselineskip]
    \small \textit{Think of the positioning parameter as adjusting the ``settings'' on \LaTeX{}'s internal  algorithms.} \\[\baselineskip]
    \column{0.4\linewidth}
    \includegraphics[width=\textwidth]{img/memes/hFloatFreeRealEstate.jpg}
\end{columns}
\begin{exampleblock}{}%{\small Figure Positioning Parameters}
\begin{table}[]
    \centering
    \small
    \def\arraystretch{1.25}
    \begin{tabular}{c l}
        h & ~~Place the float approximately \keyw{here}. \\
        \hline
        t & ~~Position at the \keyw{top} of the page. \\
        \hline
        b & ~~Position at the \keyw{bottom} of the page. \\
        \hline
        p & ~~Put on a special \keyw{page} for floats only.  \\
        \hline
        ! & ~~Override internal \LaTeX{} parameters. \\
        \hline
        H & ~~Place the float exactly \keyw{here}. (Similar to ``h!'')
    \end{tabular}
    % \caption{Figure Positioning Parameters}
\end{table}
\end{exampleblock}
\end{frame}





\begin{frame}[fragile]
\frametitle{Figures: Captioning, Labeling, and Cross-Referencing}
\end{frame}


\begin{frame}[fragile]
\frametitle{Figures: List of Figures}
\end{frame}


\begin{frame}[fragile]
\frametitle{Figures: Subfigures}
\end{frame}