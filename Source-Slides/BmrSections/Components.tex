\begin{frame}[fragile]
\frametitle{Components of a \LaTeX{} Document}
    \LaTeX{} documents have two primary components: the \keyw{preamble} and the \keyw{body}.\\[\baselineskip] \pause
    In the preamble, the author can set document geometry, load useful packages, define or redefine commands, and more. \\[\baselineskip] \pause
    The actual document content lives in the body of the document. \\[\baselineskip] \pause
    \LaTeX{} file management makes it easy to keep track of document content elements and to reorganize an entire document quickly.
\end{frame}


\begin{frame}[fragile] %{Using \texttt{minted}}
\frametitle{Components of a \LaTeX{} Document: Preamble}
\begin{block}{What is the Preamble?}
    The \emph{preamble} is everything \emph{before} the
    \verb|\begin{document}|
    command. 
    It acts as the document \emph{setup} section.
\end{block} \pause
\begin{block}{Within the Preamble:} 
    \begin{itemize}
        \item[$\bullet$] Define the document \emph{type} \pause
        \item[$\bullet$] Load \textit{packages} \pause
        \item[$\bullet$] Set up document layout, geometry \pause
        \item[$\bullet$] Define title, author, date information \pause
        \item[$\bullet$] Include user-defined macros
    \end{itemize}
\end{block}
\end{frame}


% \begin{frame}[fragile]
% \frametitle{Components of a \LaTeX{} Document: Preamble (Packages)}
    
% \end{frame}


\begin{frame}[fragile]{Using \texttt{minted}}
\frametitle{Preamble Example \#1}
\begin{minted}{latex} 
    \documentclass[12pt, letterpaper]{article}
    \usepackage{geometry}
    \usepackage{amsmath}
    \usepackage[hidelinks]{hyperref}
    \usepackage{tikz}
    \title{My first LaTeX document}
    \author{Jane Doe \thanks{Funded by xxx}}
    \date{August 2022}
    \begin{document}
\end{minted}
\end{frame}

\begin{frame}[fragile]{Using \texttt{minted}}
\frametitle{Preamble Example \#2}
\begin{minted}{latex} 
    \documentclass{beamer}
    \usepackage{url}
    \usepackage{minted}
    \usepackage{listings}
    \usepackage{enumitem}
    \usetheme{Madrid}
    \setbeamersize {
        text margin left=1cm,
        text margin right=1cm }
    \AtBeginSection[] {
      \begin{frame}
        \frametitle{Table of Contents}
        \tableofcontents[currentsection]
      \end{frame} }
    \begin{document}
\end{minted}
\end{frame}

\begin{frame}[fragile]
\frametitle{Components of a \LaTeX{} Document: Body}
\begin{block}{Body}
    The body of a \LaTeX{} document is where all of the text, equations, figures, tables, etc. live.  The body of the document begins with the command \verb|\begin{document}| and ends with the command \verb|\end{document}|.  
\end{block}
\end{frame}



\begin{frame}[fragile]
\frametitle{Components of a \LaTeX{} Document: Multiple Files}
In most \LaTeX{} documents, one has several \keytt{.tex} files --- one for each chapter or section --- joined together into a single output. \pause
\begin{block}{Input other files}
\small
\begin{minted}{latex}
\begin{document}
\input{OtherFile.tex}
\input{AnotherFile.tex}
\end{document}
\end{minted}
\end{block} \pause
\begin{itemize}[$\bullet$]
    \item \LaTeX{} compiles these as if it were one continuous file. \pause
    \item Complex \keytt{Tikz} figures or tables can be written in separate \keytt{.tex} files and loaded at the appropriate places. \pause
    \item Files that are not loaded into \keytt{main}/referenced by other files loaded into \keytt{main} will not be compiled or rendered. \pause
    \item Advantage: Easy to rearrange chapters, sections, etc.
\end{itemize}
\end{frame}

\begin{frame}[fragile]
\frametitle{Components of a \LaTeX{} Document: Summary}
    There are two primary elements of any \LaTeX{} document: the \keyw{preamble}, and the document \keyw{body}. \\[\baselineskip] \pause
    The preamble includes information about the project and its behavior (packages, geometry, user-defined information, commands, aliases, etc.) \\[\baselineskip] \pause
    The body of the document is where everything else (actual output) goes! \\[\baselineskip] \pause
    A \LaTeX{} project can include multiple files; these can be referenced internally and their order can be rearranged.
\end{frame}
